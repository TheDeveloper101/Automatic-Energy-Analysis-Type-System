%% abbrevs
\newcommand{\system}{sys-name\xspace}

%% comments
% turn off comments by uncommenting the following command and
% commenting out the second notes command
%\newcommand{\notes}[2]{} 
% turn on comments by commenting out the previous note command and
% uncommenting the following command
\newcommand{\notes}[2]{{\bf\textsf{\textcolor{#1}{#2}}}} 
\newcommand{\todo}[1]{\notes{red}{TODO: #1}}
\newcommand{\milijana}[1]{\notes{teal}{Milijana says: #1}}
\newcommand{\sai}[1]{\notes{orange}{Sai says: #1}}
\newcommand{\reminder}[1]{\notes{purple}{Reminder: #1}}
\newcommand{\mathcomment}[1]{\notes{purple}{#1}}

%command to show the full definitions (first version) or not (second version)
%\newcommand{\extra}[1]{#1}
\newcommand{\extra}[1]{}

%% fonts
\newcommand\BbbGamma{\reflectbox{\rotatebox[origin=c]{180}{$\mathds L$}}}
\newcommand\BbbGammaVar{\reflectbox{\rotatebox[origin=c]{180}{$\mathbb L$}}}
\newcommand{\m}[1]{\mathsf{#1}}
\newcommand{\mt}[1]{\mathit{#1}}
\newcommand{\ft}[1]{\textit{#1}}
\newcommand{\paragraphb}[1]{\vspace{2pt}\noindent{\bf #1}\xspace}
%% math
\newcommand{\bnfdef}{::=}
\newcommand{\bnfalt}{\,|\,}
\newcommand{\rulename}[1]{\textsc{#1}}
\newcommand{\ifthen}[3]{\m{if}\ #1\ \m{then}\ #2\ \m{else}\ #3}
\newcommand{\while}[2]{\m{while}\ #1\ \m{do}\ #2}
\newcommand{\trans}[2]{\m{transition}(#1, #2)}
\newcommand{\elet}[3]{\m{let}\ #1 = #2 ~\m{in}\ #3}
\newcommand{\eletmut}[3]{\m{let~mut}\ #1 = #2 ~\m{in}\ #3}

%% typing 
\newcommand{\tctx}{\Gamma}
\newcommand{\pctx}{\Delta}
\newcommand{\Costarrow}[1]{\stackrel{#1}{\Rightarrow}}
\newcommand{\tctxdouble}{\BbbGamma}
\newcommand{\dgamma}{\BbbGamma}
\newcommand{\lbl}{\mathbb{L}}
\newcommand{\ilbl}[1]{\scriptsize\textit{#1}}
\newcommand{\lub}{\sqcup}
\newcommand{\entails}{\models}
\newcommand{\ok}{\m{ok}}
\newcommand{\subtp}{\sqsubseteq_t}


%% algo

%algo stuff 
\usepackage{eqparbox}
\newdimen{\algindent}
\setlength\algindent{1.5em} 
\algnewcommand\LeftComment[2]{%
\hspace{#1\algindent}$\triangleright$ \eqparbox{COMMENT}{#2} \hfill %
}

%% fonts
\newcommand{\alg}[1]{\ensuremath{\mathit{#1}}}


%% configurations
\newcommand{\prog}{\ft{prog}}
\newcommand{\seg}{\ft{seg}}
\newcommand{\dom}{\m{dom}}
\newcommand{\stepsto}{\longrightarrow}
\newcommand{\Stepsto}[1]{\stackrel{#1}{\Longrightarrow}}
\newcommand{\SeqStepsto}[1]{\stackrel{#1}{\longrightarrow}}
\newcommand{\MStepsto}[1]{\stackrel{#1}{\Longrightarrow^*}}
\newcommand{\MSeqStepsto}[1]{\stackrel{#1}{\longrightarrow^*}}
\newcommand{\proj}[2]{#1|_{#2}}
\newcommand{\stepstopc}[1]{\stackrel{#1}{\rightarrow}}

%% proofs
\newcommand{\ee}{\mathcal{E}}
\newcommand{\de}{\mathcal{D}}

\newcommand{\denote}[1]{[\![ #1 ]\!]}

\newcommand{\xstep}[1]{\xmapsto{#1}}
\newcommand{\xsteps}[1]{\xmapsto{#1}^*}
\newcommand{\step}{\mapsto}
\newcommand{\steps}{\mapsto^*}

\newcommand{\erase}[1]{{#1}^-}

\newcommand{\entry}[1]{\ft{Entry}(#1)}

\newcommand{\sametraceequiv}{\prec}
\newcommand{\twotraceequiv}{\approx}
\newcommand{\modeequiv}[1]{\approx_{#1}}
\newcommand{\outputequiv}{\approx^\m{out}}
\newcommand{\inputequiv}{\approx^\m{in}}



%% theorem environments

%%% proofs
\newif\ifproofs
%%%%%% uncomment to not show proofs 
%%\proofsfalse
%%%%%% uncomment to show proofs 
\proofstrue

\newtheorem{thm}{Theorem}
\newtheorem{lem}[thm]{Lemma}
\newtheorem{cor}[thm]{Corollary}

\newtheorem{defn}[thm]{Definition}
\newtheorem{assumption}[thm]{Assume}
\newenvironment{proofsketch}{\vspace{-1pt}\textsc{Proof (sketch). }\hspace*{0.25em}}{ \hspace*{\fill} \qed}

\newcommand{\myparagraph}[1]{\vspace{2pt}\noindent{\bf #1}}
