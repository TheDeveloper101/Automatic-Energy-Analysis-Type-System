\documentclass{article}

\usepackage{amsmath,amssymb,amsthm,mathtools}
\usepackage{mathpartir}
\usepackage{geometry}

\title{Automatic Energy Resource Analysis Typesystem}
\author{Sai Divvela}
\date{October 2025}

\begin{document}
Language:
\[
\begin{array}{rcll}
e & ::= & x & \text{(variable)} \\
  & \mid & \text{let } x = e_1 \text{ in } e_2 & \text{(let)} \\
  & \mid & \text{transition}(x, \text{st}) & \text{(transition)} \\
  & \mid & e_1\ e_2 & \text{(application)} \\
  & \mid & \lambda x.\,e & \text{(function abstraction)} \\
  & \mid & \text{share } x \text{ as } x_1, x_2 \text{ in } e & \text{(share)} \\
  & \mid & \text{atomic}(e) & \text{(atomic)} \\
  & \mid & e_1;\ e_2 & \text{(seq)} \\
  & \mid & \text{if } e \text{ then } e_1 \text{ else } e_2 & \text{(branch)}
\end{array}
\]
\\

$(\Gamma:x \rightarrow \tau,f \rightarrow \left( \tau_{1} \rightarrow \tau_{2} \right)$\\
\(S ::=\) (\(\langle s_{0},s_{1},\ldots,s_{n}\rangle, <\))\\
cost(\(s_{i},s_{j}\)) = transition cost between \(s_{i},s_{j}\) when
\(s_{i} \leq s_{j}\)\\
cost(\(s_{i}\)) = cost of being in state \(s_{i}\)\\
\(\Delta:p \rightarrow (S,s)\) where \(s\) is the current state of the
peripheral and \(S\) is the state lattice associated with the
peripheral.

\(\Gamma,\Delta, \vdash e:\tau\overset{C}{\rightarrow}\Delta'\)\\
\(\tau ::= b~|~\tau_{1} \rightarrow \tau_{2}~|~\rho\)\\
\(s_{i} \sqcap s_{j} = s_{j}\) if \(s_{i} < s_{j}\)\\
\(\Delta' \sqcap \Delta'' = \Delta^{\ast} \rightarrow \forall p \in \Delta',\Delta'',\Delta^{\ast (p)} = \Delta'(p) \sqcap \Delta''(p)\)\\
In words, the join of two peripheral contexts is a new context whose
elements are the join of the current state of each peripheral in the
context.\\
We also assume that every lattice has a unique \(\top\) element that is
the least upper bound of all the elements.

\begin{mathpar}
\inferrule*[right=T-Seq]
{
  \Gamma, \Delta, s_{\max} \vdash e_1 : \tau_1 \xrightarrow{C_1} \Delta', s'_{\max} \\
  C_{1\text{-max}} = \mathrm{cost}(s_{\max}) \\
  \Gamma, \Delta', s'_{\max} \vdash e_2 : \tau_2 \xrightarrow{C_2} \Delta'', s''_{\max} \\
  C_{2\text{-max}} = \mathrm{cost}(s'_{\max}) \\
  C = C_1 + C_2 + C_{1\text{-max}} + C_{2\text{-max}}
}
{
  \Gamma, \Delta, s_{\max} \vdash e_1; e_2 : \tau_2 \xrightarrow{C} \Delta'', s''_{\max}
}

\inferrule*[right=T-Atomic]
{
  \Gamma, \Delta, s_{\max} \vdash e : \tau \xrightarrow{C_1} \Delta', s'_{\max} \\
  C_{\max} = \mathrm{cost}(s'_{\max}) \\
  C = C_1 + C_{\max}
}
{
  \Gamma, \Delta, s_{\max} \vdash \text{atomic}(e) : \tau \xrightarrow{C} \Delta', s'_{\max}
}

\inferrule*[right=T-Branch]
{
  \Gamma, \Delta \vdash e : \text{bool} \xrightarrow{C'} \Delta \\
  \Gamma, \Delta \vdash e_1 : \tau \xrightarrow{C_1} \Delta_1 \\
  \Gamma, \Delta \vdash e_2 : \tau \xrightarrow{C_2} \Delta_2 \\
  C = C' + \max(C_1, C_2) \\
  \Delta' = \Delta_1 \sqcap \Delta_2
}
{
  \Gamma, \Delta \vdash \text{if } e \text{ then } e_1 \text{ else } e_2 : \tau \xrightarrow{C} \Delta'
}

\inferrule*[right=T-Transition]
{
  \Gamma, \Delta \vdash x : \rho \\
  \Delta(x) = s_i \\
  s_i \le s \\
  C_t = \mathrm{cost}(s_i, s)
}
{
  \Gamma, \Delta \vdash \text{transition}(x, s) : \tau \xrightarrow{C_t} \Delta[x : s]
}

\end{mathpar}
\end{document}
